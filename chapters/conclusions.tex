\chapter{Conclusions}
\label{chap:conclusions}

In this final chapter, we will summarise the work and results presented in this thesis. We will discuss the main contribution of this thesis and the significance of the results and insights gained, while also highlighting some open issues. Finally, we will give an outlook on future directions for further research into the logical and cognitive aspects related to justifications for entailments of OWL ontologies.

\section{Summary of contributions}

In summary, this thesis introduced the notion of justificatory structure, provided definitions for the different aspects of structure, proposed strategies for exploiting the justificatory structure of an ontology in order to provide improved debugging support, and analysed a set of ontologies from the bio-medical domain to determine the prevalence and extents of structural phenomena in OWL ontologies used in practice.

\subsection{Design decisions for finite entailment sets}

We first discussed the issue of limiting the infinite set of entailments of an ontology to a finite entailment set and proposed several design decisions to be made in order to arrive at a sensible \emph{representation} of a finite entailment set. These design decisions were largely motivated by the issue of \emph{counting} entailments, e.g.\ for the purpose of comparing the inferential power of two ontologies, which requires the number of entailments to grow monotonically when adding axioms to the ontology. 

We found that certain design decisions cause the number of entailments to grow non-monotonically: not including asserted axioms or indirect subsumptions in a finite entailment set, for instance, can lead to \emph{fewer} entailments in the set if we add axioms to an ontology. We also provided a practical definition for distinguishing imported, mixed, and native entailments based on their justifications. Finally, we provided a shorthand notation for referring to a specific representation of a finite entailment set and demonstrated the effect of different design decisions toy examples and real-life applications of finite entailment sets of OWL ontologies.


\subsection{Justificatory structure and justification isomorphism}

In Chapters 4 and 5, we introduced the notion of justificatory structure using a graph representation of the justifications, justification axioms, and entailments for a given set of entailments and justifications in an ontology. Structural aspects include graph metrics such as the in- and out-degrees of justification and axiom nodes, graph components, and overlap of varying degrees. Overlap between justifications, that is, shared axiom sets, are of particular interest in the context of ontology debugging, as they potentially lead to smaller repairs, while also helping the understanding of multiple justifications through lemmas.

We then moved on to discuss the issue of structural similarities between justifications and extended the existing notion of justification isomorphism to two new equivalence relations, subexpression-isomorphism \sisom and lemma-isomorphism \lisom. Under (strict) isomorphism, we consider two justifications to be equivalent if they use the same constructors and axiom types and only differ in the class, property, and individual names they use. Subexpression-isomorphism extends this notion to cover justifications which differ in the expressions they use if those expressions can be substituted by freshly generated variable names; that is, if their complex subexpressions are used in a \emph{propositional} way. Lemma-isomorphism applies the principle of subexpression-isomorphism to a lemmatised justifications, that is, a justification in which a subset $S$ was replaced with an entailment $\lambda$ of $S$. 

The two new notions of isomorphism were designed to be \enquote{upwards-compatible}, that is, (strictly) isomorphic justifications are subexpression-isomorphic, and subexpression-isomorphic justifications are considered to be lemma-isomorphic. We showed that subexpression-isomorphism is indeed an equivalence relation under certain (non-obvious) side conditions, and provided a proof for the transitivity of subexpression-isomorphism.

\subsection{Reducing user effort}

In Chapter 6, we looked at the problem of debugging erroneous entailments in OWL ontologies, providing a definition for \enquote{debugging problems} and how to determine whether a modification successfully solved the debugging problem. We constructed a simple model for measuring the \emph{effort} involved in solving a debugging problem using justification-based explanation tools. This model also allows us to measure whether a debugging strategy \emph{reduces} the effort involved in debugging a set of entailments by introducing an \emph{alleviation factor} $a$. 

We proposed several strategies for exploiting the justificatory structure of an ontology in order to reduce the user effort when faced with multiple justifications. These strategies include presenting the user with high-ferquency axioms, enriching justifications with common lemmas originating from overlaps, and presenting the user with an abstract justification template of isomorphic justifications.

\subsection{Experimental results}

Finally, we presented a survey of OWL and OBO ontologies from the NCBO BioPortal. We found that the majority of justifications for direct and indirect atomic susbumptions between satisfiable classes can be classified as \enquote{atomic subsumption chain} justifications, whereas the number of \enquote{complex} justifications found across the corpus is comparatively low. For those entailments that \emph{do} have complex justifications, the number of justifications per entailment was found to be surprisingly high, with an average of 8 justifications per entailment  and 70\% of the entailments in the corpus having \emph{multiple} complex justifications. Shared axioms and axiom sets were found to occur frequently across the corpus: Over 73\% of the justifications in the corpus were derived from some other justification (which may be partially due to the inclusion of indirect subsumptions in the entailment set). Arbitrary overlap between justifications for \emph{single} entailments is highly prevalent in the corpus, with the majority of overlaps having a size of around 5 axioms and occurring in around 5 justifications. 

Our analysis of justification isomorphism in the BioPortal corpus revealed that justification isomorphism between justifications for \emph{individual} entailments occurs fairly frequently: Strict isomorphism applied to justifications for individual entailments causes an average reduction from justifications to templates by 33\%, whereas subexpression-isomorphism has only marginal effects in some ontologies. However, if we consider the logical diversity of all justifications for all entailments in the corpus, strict isomorphism and lemma-isomorphism have a significant impact: the 141,560 justifications in the corpus are effectively reduced to just over 12,500 templates (a reduction of 91.2\%) by strict isomorphism, and lemma-isomorphism finally reduces this number to only 5,487 templates. More strikingly, we have found that 75\% of the justifications in the corpus can be covered by only 277 of the most frequent templates for lemma-isomorphism. Across the set of laconic versions of the justifications in the corpus, the \emph{relative} effects of the three isomorphism types are roughly the same; however, the final number of templates for lemma-isomorphism is considerably lower at only 1,789 templates. This shows that the logical diversity of justifications is far lower than their material manifestiations, as removing all superfluous parts in a corpus of 141,560 regular (complex) justifications reduces it to just over 1\% of its original size.



\section{Significance of results}

Against the backdrop of justification-based debugging support, this thesis has advanced the state of knowledge we have of the relations between justifications in OWL ontologies. It highlighted possible new strategies for generating OWL ontology metrics and for providing improved debugging support, which lays the foundations for future approaches to building user-friendly and efficient  OWL ontology tools for both ontology analysis and ontology development.

First, the design decisions we provided for generating and representing finite entailment sets draw from the multitude of modifications and \enquote{hacks}, as well as misunderstandings we have encountered in OWL tools and analytical applications. Previously, there has been no clear account of the various factors that have an impact on the size of entailment sets and no convenient way of referring to a certain type of entailment set, which we believe to have rectified with the design decisions and shorthand notation provided in this thesis. 

This has been the first in-depth investigation of the justificatory structure of OWL ontologies. In our survey of the BioPortal ontologies, we have found that, while a large number of entailments only have trivial atomic subsumption chain justifications, a significant number of entailments indeed has multiple complex justifications. This finding shows that improved debugging support for dealing with multiple justifications is clearly necessary, as users have a high chance of encountering an entailment that has multiple non-trivial justifications.

While root and derived justifications and axiom frequency have been (somewhat implicitly) used in justification-based debugging tools, there has been no extension of these relations to cover arbitrary overlap. There have been some previous experiments regarding root and derived relationships between justifications \cite{kalyanpur05mi,meyer10vd}, however, the survey presented in this thesis is the first large-scale experiment investigating the occurrence of both root and derived justifications and arbitrary justification overlap in OWL ontologies. We have found that all types of overlap occur frequently, which provides us with important knowledge of the structural relations between justifications and informs future ontology debugging tools, which can make use of these structural aspects.

Thus far, justification isomorphism has only been mentioned in the context of sampling justifications for a user study \cite{horridge11gj}. The results presented in our survey of justification isomorphism in the BioPortal corpus confirms that a) isomorphic justifications can be determined in practical time even using a naive implementation, and b) a large number of justifications are structurally isomorphic. This shows that template-based debugging support is both feasible for and widely applicable to justifications found in practice.

However, we have also found that the newly introduced relations, subexpression-isomorphism and lemma-isomorphism, do not have as big an impact on the BioPortal justifications as we would have hoped. S-isomorphism in particular only affects a fraction of the justifications in the corpus, whereas lemma-isomorphism occurs in most ontologies, but only in small numbers. This implies that, for most justifications, strict isomorphism may already be sufficient for finding a common template. On the other hand, the small effect of s-isomorphism also shows us that there exists \emph{real} logical diversity in the modelling of ontologies, and that complex subexpressions are generally \emph{not} used in a propositional way. Furthermore, while we could successfully prove the transitivity of s-isomorphism, the transitivity of l-isomorphism and the selection of suitable lemmatisations remains an open question. In order to make l-isomorphism useful in OWL applications, we need to further explore potential lemmatisations which are guaranteed to preserve the transitivity of l-isomorphism.

With the expection of root and derived justifications and a focus on minimal repairs, most of the existing justification-based debugging techniques do not consider the issue of multiple justifications for repair and treat justifications as isolated entities. By suggesting structure-based coping strategies for multiple justifications we have made a step towards improved debugging support which is targeted at multiple justifications. The introduction of a model for measuring \emph{and quantifying} the success of a coping strategy in particular paves the way for principled  empirical research into the effects of different justification-based debugging techniques.

%Note that the work presented in this thesis considers only the syntax and semantics of OWL justifications and does not pay attention to the domain knowledge that is being modelled in an ontology. It is clear to see that domain knowledge and a good understanding of the OWL syntax and semantics may in many cases already be sufficient for users to find a suitable repair; in fact, from anecdotal evidence we know that some users rely entirely on their domain knowledge to repair erroneous entailments. However, we argue that improved debugging facilities are essential to OWL development tools, as a) they provide necessary support when relying on domain knowledge is not sufficient (e.g.\ in the case of the \emph{Movie} ontology which models fairly simple common knowledge but has a \emph{logically} hard justification), and b) it can support the user even if they \emph{could}  resolve the error without additional support, thus potentially speeding up the debugging process.

\section{Future directions}

While we have covered a broad range of topics in this thesis, it is clear that there is plenty of room for future work. In this section we will outline potential extensions of this work, which cover the theoretical foundations of justificatory structure, further experiments, as well as applications of the strategies proposed in this work.

\paragraph{Lemma-isomorphism}
We have made some progress towards defining lemma-isomorphism and finding suitable lemmatisations; however, there are still some open questions remaining. First of all, we have restricted the lemmatisations to (maximal) atomic subsumption chains in order to demonstrate the concept of l-isomorphism, which has already had some visible effects. On the other hand, if we think back to the original motiviation for l-isomorphism---the \emph{Pizza} ontology in which there exists several similar reasons for why some \cn{Pizza} is a subclass of \cn{InterestingPizza}---we can see that atomic subsumption chain lemmatisations do not cover the justifications we can find there. Thus, extending the set of lemmatisations to be used in l-isomorphism based on their \emph{obviousness} for OWL users seems to be an important next step, in particular since we have shown  that over three quarters of the justifications in the test corpus could be covered by a strikingly small number of templates for lemma-isomorphism. Adding only a small number of additional lemmatisations may already be sufficient to cover the vast majority of justification shapes found in OWL ontologies.

However, since we have seen that preserving the transitivity of our isomorphisms is rather non-trivial, this will also require a thorough investigation of the conditions lemmatisations have to meet in order for l-isomorphism to be transitive.


\paragraph{Dealing with masking and superfluity}

While we discussed issues such as non-laconic justifications and justification masking where applicable, our introduction to justificatory structure and the survey of the BioPortal ontologies did not fully explore the effects of masking and superfluity. We know that masking and superfluity are frequent occurrences in OWL ontologies \cite{horridge11ab} and that superfluous parts may cause users difficulties in understanding justifications \cite{horridge11gj}. In our analysis of isomorphism we also found that a large number of justifications are considered to be non-isomorphic due to them containg superfluous parts, and that laconicising justifications significantly reduces the overall diversity of justifications. A more in-depth comparison between the justificatory structure of non-laconic and laconic justification sets will help us gain a better understanding of how these phenomena affect the justificatory structure of an ontology, and to which extent this has an impact on the debugging process.


\paragraph{Effects on justification computation}

Another open question that arises from the work presented in this thesis is the effect of justificatory structure on justification computation, and, in a wider sense, reasoning performance. While we took a brief glance at the effects of graph components on the performance of justification computation, we omitted an in-depth investigation of the relations between structure, hitting set tree size, calls to a \enquote{find one} subroutine, and computation times. This will require further experiments and artificial generation of justifications to test the effects of various degrees of justification overlap, activity, and graph structure in isolation.  Further insights into the relation between justificatory structure and justification computation would potentially allow us to generate guidelines, ontology design patterns, or even automated tool features to yield easily computable justifications. Such guidelines might either be in line with existing ontology design patterns, or potentially require users to sacrifice some aspects of \enquote{good modelling} for the sake of easy justification computation; thus, an investigation of  the interplay between existing ontology design patterns and justificatory structure is another aspect of justificatory structure worth exploring.

Further, since justifications are, in some sense, responsible for the subsumption relationships in an ontology and therefore its class hierarchy, we may also ask: how does the justificatory structure of an ontology affect the classification performance of an OWL reasoner? We have made some steps into this direction with JustBench \cite{bail10wb}, a justification-based micro-benchmarking tool for OWL reasoners; however, in JustBench we only used justification-entailment pairs in order to measure the performance of entailment checking on naturally arising ontology subsets, which did not involve any analysis of justificatory structure.


\paragraph{Visualisation and user interaction}

Possibly one of the most intriguing directions for future work is the exploration of user interaction mechanisms to exploit the strategies presented in this thesis for implementation in OWL tools. We have made a first attempt at outlining interaction mechanisms for multiple justifications in Chapter \ref{chap:understanding}. However, the design and implementation of a \emph{useful} OWL debugging tool will require significantly more research into the cognitive aspects of how users read and understand OWL constructs and axioms, starting with the fundamental question of whether OWL users build \emph{mental models} \cite{johnson-laird80kt} of the information they digest, or whether they apply \emph{rules} and symbol manipulation in order to understand reasoning processes, and to which extent this behaviour depends on the user \emph{profile} and their specific \emph{task}. While we have gained some insights in recent years into how people process OWL and what causes them difficulties in understanding, we are far from having a clear picture of the cognitive processes associated with interacting with OWL ontologies.

Once we have a better understanding of these cognitive processes as well as the demands and requirements of OWL users with different backgrounds, we need to identify a suitable approach to visualising and interacting with OWL ontologies and justifications. There have been various visualisation tools for (particular aspects of) OWL ontologies, such as OWLViz\footnote{\url{http://www.co-ode.org/downloads/owlviz/}} and CropCircles \cite{wang06wr} which visualise the class hierarchy of an OWL ontology, SuperModel \cite{bauer09ru} which displays segments of a model of the ontology, and DeMost \cite{del-vescovo11aa}, which represents the atomic decomposition of an ontology as a graph of axioms and axiom sets. 

However, none of these approaches (with the exception of CropCircles perhaps which uses a non-standard rendering of nested circles to represent subsumption relationships) seem to have grown out of a focussed investigation of the suitability of different interaction mechanisms for the task at hand; we may argue that graph-based approaches were chosen because they are simply the most straightforward way of representing the types of relations occurring in an OWL ontology. Given the progress made in interactive data visualisation tools, such investigations would be both highly interesting and worthwhile, as improved tools for OWL ontology building and debugging may as well contribute to the wider acceptance and propagation of OWL.


\paragraph{Applications of justifications}

Finally, another topic we have only touched upon in this thesis is the application of justifications beyond debugging and repair. One of these areas is ontology comprehension and learning, that is, OWL users wanting to familiarise themselves with an unknown ontology, or OWL novices attempting to understand the implications of reasoning in OWL ontologies. We have seen that many aspects of justificatory structure, such as justification overlap and isomorphism, reveal deeper insights of the interactions between axioms in an ontology, while also providing information akin to ontology \emph{patterns}. We can imagine using aspects of justificatory structure to teach and learn \enquote{reasoning patterns} in OWL ontologies, which may support OWL novices in building ontologies. 

Further, the justificatory structure of an OWL ontology, such as the number of justifications, the number and sizes of overlaps, and the structural similarities between justifications, provides us with \emph{implicit} ontology metrics. The number of justifications per entailment has already been used as an ontology metric to determine the \emph{justification richness} of an ontology \cite{mikroyannidi11aa}, and we have shown how two seemingly similar ontologies with approximately the same number of classes and axioms can differ vastly in their justificatory structure \cite{bail11jm}. Integrating justification-based metrics into OWL tools will make these differences visible to OWL developers, allowing them to compare and rank ontologies based on an extensive set of both explicit and implict metrics.



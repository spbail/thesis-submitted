The Web Ontology Language OWL is based on the highly expressive description logic \dl{SROIQ}, which allows OWL ontology users to employ out-of-the-box reasoners to compute information that is not only explicitly asserted, but \emph{entailed} by the ontology. \emph{Explanation} facilities for entailments of OWL ontologies form an essential part of ontology development tools, as they support users in detecting and repairing errors in potentially large and highly complex ontologies, thus helping to ensure ontology quality. 

\emph{Justifications}, minimal subsets of an ontology that are sufficient for an entailment to hold, are currently the prevalent form of explanation in OWL ontology development tools, and they have been found to significantly reduce the time and effort required to debug erroneous entailments. A large number of entailments, however, have not only one but \emph{many} justifications, which can make it considerably more challenging for a user to find a suitable repair for the entailment.

In this thesis, we investigate the relationships between multiple justifications for both single and multiple entailments, with the goal of exploiting this \emph{justificatory structure} in order to devise new \emph{coping} strategies for multiple justifications. We describe various aspects of the justificatory structure of OWL ontologies, such as shared axiom cores and structural similarities. We introduce a model for measuring user effort in the debugging process and propose debugging strategies that exploit the justificatory structure in order to reduce user effort. Finally, an analysis of a large corpus of ontologies from the biomedical domain reveals that OWL ontologies used in practice frequently exhibit a rich justificatory structure.

\bigskip 

\noindent Date of submission: 26/04/2013